%\documentclass[10pt,twocolumn]{article}

%\usepackage{graphicx, color}
%\usepackage[font=small,labelfont=bf]{caption}
%\usepackage[subrefformat=parens]{subcaption}
%\usepackage{amsmath}
%\usepackage{subcaption}
%\usepackage{url}
%\usepackage{listings}

%\begin{document}

%\title{Finding An Unified Programming Model for Future Network Functions}

%\maketitle

%\section{Background and Motivation}

In recent years, the research community has witnessed the quick development of
network function virtualization (NFV). DPDK \cite{dpdk} and Netmap
\cite{rizzo2012netmap} use kernel bypassing to speed up the performance of NF
software. They have become the default libraries for implementing high-speed
modern NF software. NFV management systems such as E2 \cite{palkar2015e2} are
built to dynamically scale virtual instances running different NFs. NFs are
augmented with fault tolerance \cite{sherry2015rollback} and flow migration
\cite{gember2014opennf} to improve the failure resilience.

However, despite all these advancements, a core problem is not well-solved by
existing work: what should be the default programming abstraction for implementing NF
software, so that the diverse requirements of NF software can be well-captured
by this abstraction? To show the importance of this problem, let me first discuss
the diversity of NF software.

\subsection{Diversity of NF Software}


\noindent \textbf{Simple Packet Processing Program.} Example NFs include
firewall, NAT and load balancer. The word ``simple'' actually means that the way
that these NFs manipulate packets is simple: they take an input packet,
perform necessary packet transformation and book-keeping, then they release the
packet to the outside. Taking NAT as an example. After receiving an input
packet, NAT may update the connection status associated with the flow, then the
NAT performs an address translation to substitute the IP address and port of the
packet. Finally, NAT sends the packet out from the output port.

\textit{These NFs can be effectively implemented inside a polling loop and can be
seamlessly integrated with either DPDK or Netmap for maximum performance.}

\noindent \textbf{NFs with Intensive File I/O.} Example NFs include PRADs
\cite{prads} asset monitor and Snort \cite{snort} intrusion detection system
(IDS). For instance, PRADS is a passive real-time asset detection system, which
listens to network traffic and logs important information on hosts and services
it sees on the network. This information can be used to map the underlying
network, letting network operators know what services and hosts are active, and
can be used together with IDS/IPS setup for "event to application" correlation.

Both PRADs and Snort can be ported to use DPDK to speed up packet processing
\cite{201546}. Even after porting to DPDK, both NFs fail to achieve 10Gbps line
rate processing \cite{201546}. The primary reason for this undesirable number is
due to logging. After porting to DPDK, the worker threads of both NFs keep
polling for new packets and maintain CPU usage to 100\%. But when both NFs log
important events, they have to access system calls related to file system
processing, generating expensive context switches and compromising the packet
processing throughput.

\textit{These NFs can be accelerated using DPDK and Netmap, but they still need
  to step into the kernel to log events to the files.} NFs with intensive file
I/O remain to be interesting phenomena in existing NFV research. People have
strived to remove context switches associated with kernel networking stack by
bypassing the kernel with DPDK, but they fail to remove the context switches
associated with kernel file systems during logging.

\noindent \textbf{NFs with Reliable Communication to External Services.}
Example NFs include S-CSCF in IMS system \cite{3gpp-ims} and NFs that need to
replicate their states on back NFs.

S-CSCF is an important middlebox sitting at control plane of the IMS system. It
processes SIP \cite{sip} messages by contacting several external
services. Taking the S-CSCF implementation of a famous open source IMS project
Clearwater \cite{project-clearwater} as an example, when processing SIP messages,
S-CSCF needs to log SIP registration information on a Memcached \cite{memcached}
cluster and acquire user information by querying a dedicated storage server
called Home Subscriber Server (HSS). The S-CSCF implementation of Clearwater
uses kernel TCP/IP stack to carry out reliable communication to all the required
external services, seriously limiting the maximum throughput that S-CSCF can
achieve. Our experience with Clearwater shows that a single worker thread in
S-CSCF can only process SIP messages with the bandwidth of 40Mb.

FTMB \cite{sherry2015rollback} is the state of art system for NF replication. It
employs a primary-backup replication strategy. On the primary NF instance,
after each packet is processed, the packet is passed to the backup over a
reliable communication channel for replication. We can treat the replication
process as communicating external services: each input packet processed by the
primary instance must be reliably delivered to the backup instance. FTMB uses
DPDK to speed up packet processing and implements its own reliable communication
channel on top of DPDK. But the implementation detail of the reliable
communication channel is omitted from the paper. It would be desirable to
implement the reliable communication channel using a user-level TCP/IP stack
like mTCP \cite{179773}, so that the performance of FTMB is stable (a
handcrafted reliable communication channel may be unstable and lack of flow
control) and it is easier to reproduce FTMB implementation for both academic and
industrial usage. \textit{However, without a good programming abstraction,
  integrating a user-level TCP implementation like mTCP with replication
  strategy like FTMB is not a trivial task:} mTCP exposes an event-driven
programming interface like Linux epoll. The application thread using mTCP does
not sit in the same thread as the mTCP worker thread. But FTMB requires that the
same worker thread handles both NF packet processing and reliable communication
to ensure correct replication.

\textit{Some of these NFs abandoned DPDK
  and Netmap, use kernel networking stack to provide reliable communication
  channel, but sacrifice performance. Some of these NFs use DPDK and Netmap to
  speed up packet processing and implement their own reliable communication
  channel, but sacrifice the stable performance and flow control provided by TCP/IP. }



\noindent \textbf{NFs that Process Events Raised by Lower-level System
  Components.} Example NFs include Snort IDS \cite{snort} and Bro IDS
\cite{bro}. The two IDSes alert potential attacks by matching the flow protocols
and analyzing flow payloads with an automaton. They can be decoupled into two
parts: A low-level system is responsible for re-assembling the TCP stream and
generating events associated with the TCP stream, i.e. connection setup,
packet re-transmission, and the new packet payload. A high-level event driven
system is responsible for reacting to the events raised by the low-level system,
i.e. in the case of a fake re-transmission forged by an attacker, the IDS drops the
flow and raises an alert. These IDSes can be effectively accelerated using mOS
\cite{201546}, which substitute the low-level system that raises flow-related
events. mOS is accelerated using DPDK and is an improved version of mTCP \cite{179773}.

\textit{The low-level system of these NFs can be accelerated with DPDK. However,
  the low-level system like mOS is usually targeted to process TCP/IP protocol
  and can not be extended to process non-TCP/IP protocol.}

\noindent \textbf{Summary.} Now we briefly discuss the similarities and
differences of all the discussed NF software.

\noindent \textbf{Similarity.} Most of these NFs can be accelerated with DPDK or
Netmap (except for S-CSCF, which relies on kernel networking stack, but we can
still accelerate it by porting it to user-level TCP/IP stack like mTCP). Using
DPDK or Netmap means that the worker threads in these NFs become busy polling
thread that keeps CPU usage to 100\%, implying that any system calls entering
the kernel context may compromise the performance of these NFs.

\noindent \textbf{Difference.} These NFs have different working goals and
operate at different levels. Simple packet processing programs only manipulate
raw packets. They do not rely on any external services. PRADs needs to do file
I/O. FTMB and S-CSCF need to communicate with external services. Snort and Bro
operate on a high-level that reacts to flow-level events raised by a low-level
system components. These differences lead to diverse implementation details,
making it hard to find an appropriate abstraction to unify these NFs. 

\subsection{One Abstraction to Rule Them All}

Just like the dedication that physicists put into the grand unified theory,
computer scientists also have been searching for a unified programming
abstraction that can capture a variety of applications. In terms of NFV, if a
unified programming abstraction can be found for all the NFs mentioned in
the previous section, programmers can enjoy the following benefits.

First, by optimizing the performance of the library that provides the
unified programming abstraction, we can improve the performance for a huge variety of
NFs. There is no need to optimize each NF, which might take a huge amount of
labor work.

Second, ease NF software development. Once the implementor becomes familiar with
the programming abstraction, he is able to create different types of NFs without
learning different programming paradigms or constructing different libraries.

Finally, it makes important research and industrial result easily reproducible,
as the unified programming abstraction makes people play on the same ground.

Such a programming abstraction is readily accessible for NFV implementors and
researchers, which is functional reactive programming, especially the subset
related to futures, promises and continuations.

\subsection{Futures and Promises}

Futures and promises are important terminologies in functional reactive
programming. A future represents a value that is going to be computed while a
promise represents the action when the computation is done. This simple
programming paradigm can easily capture most of the asynchronous programming
patterns. Let me briefly explain how futures and promises can be mapped to NFs
discussed in previous sections.

For simple packet processing program, the futures are packets that are going to
be received whereas the promises are packet handler functions.

For PRADs, the futures and promises can be combined to implement efficient file
system logging. The futures represent the logging action that will log events
raised by PRADs to the file system. The promises represents post actions when
the logging is done.

For S-CSCF and FTMB, futures and promises can be used to implement an efficient
user-space TCP/IP stack. The futures are still packets to be received, but the
promises become TCP/IP stack handlers.

For Snort and Bro, futures and promises can be used to implement a low-level
system that raises flow events. Futures flow events that are going to be raised,
promises are event handlers for these events.

The future-promise programming abstraction can be efficiently implemented with
a small runtime overhead (i.e. asynchronous C++ library Seastar
\cite{seastar}). The programming abstraction can fully bypass the entire kernel,
even in terms of file logging (with the help of DMA), providing satisfactory
performance for modern NF software.

\subsection{Contribution}

In this paper, we are going to make the following contributions.

First, we are the first to apply functional reactive programming as a generic
method for building a variety of NF software. We use seastar as the underlying
library for providing the reactive programming abstraction.

Second, we carry out case studies to show how functional reactive programming
can be used to construct 4 different types of NFs, with diverse requirements.

Finally, we show that the performance and ease of implementation are greatly
improved by using functional reactive programming. In particular:

\noindent \textbf{We re-implement PRADs using functional reactive programming.}
The resulting PRADs is capable of logging to file system at a throughput of
several gigabits per second.

\noindent \textbf{We create a new primary-backup replication strategy.} The new
primary-backup strategy is capable of processing packets at line rate. The
biggest difference between this replication strategy with FTMB is that it does
not need to checkpoint the master NF instance, greatly simplifying the
implementation effort. It has no replay time and introduces no extra latency
caused by checkpointing.

\noindent \textbf{We re-implement mOS using our new programming abstraction.} We
also port PRADs to use the new mOS and show that the new PRADs can be several
times faster than that in the mOS paper \cite{201546}.


%\section{Primary-backup NF Replication Without Rolling Back}
%To be continued.

%\input{section/second.tex}
%%\section{Asynchronous Programming}
%The core idea of asynchronous programming is to handle all the tasks that need
%to block waiting for results asynchronously, so that nothing is really blocked.

%It can be implemented using an active poll loop. The active poll loop polls
%different event source for events. Whenever an event happens, the poll loop
%executes a callback function associated with the event to handle the event.

%An improved asynchronous programming style is to use futures and promises. The
%future and promise abstraction can handle asynchronous programming in a nice
%manner.

%Asynchronous programming is the solution for all the problems that I have
%mentioned in the previous section.

%\section{Seastar and OSv}

%Seastar is a modern asynchronous programming library with future and promise
%style. It can be combined together with DPDK to achieve low-level packets
%processing. It also includes a user-space TCP/IP stack that is driven by
%asynchronous programming.

%It is the best asynchronous programming library that we can use to build network
%functions.

%Seastar can also run in OSv, a unikernel. In this way we can achieve
%high-performance virtualization.

%\section{Planned Roadmap}

%I plan to do the following case studies using Seastar in this work.

%First, implementing an unified NFV platform like NetBricks. Highlight that with
%the help of C++ unique\_ptrs, we can achieve the similar software memory safety
%like NetBricks.

%Second, port PRADS to use Seastar. Show that with the help of asynchronous
%programming, we can greatly improve the performance of PRADS.

%Third, implement a simplified version of mOS. Show that asynchronous programming
%can greatly simplify flow event generation and processing.

%Fourth, re-implement FTMB. Show that how easy it is to implement primary-backup
%replication for network functions using seastar.

%Fifth, a simplified SDN switch using Seastar, show it's performance boost when
%compared against OpenVSwitch.

\section{Promises and Cooperative Threads}

In this section, I will first give a detailed overview about promises and
cooperative threads used by Seastar fraemwork. Then I will introduce how to
apply promises and coopeartive threads to NFV.

\subsection{Lwt}

The core building blocks of Seastar are promises and cooperative
threads. Clearly, these fancy concepts come from the world of functional
programming languages. There is an Ocaml library called Lwt \cite{vouillon2008lwt}, which also
implements promises and cooperative threads. In the following sections, I will
first discuss how promises and coopeartive threads are implemented in
Ocaml. Then I will show how they are implemented in Seastar.

\subsubsection{Lwt Overview}

When wring programs like network servers, non-blocking is a very important
property that ensures the runtime efficiency of the network servers. There are
two dominant techniques to make the program non-blocking. The first one uses
multi-threading to support non-blocking, i.e. whenever a blocking operation is
going to be made, the main thread of the program lanuches another thread to
handle the blocking operation. The second one uses event-based programming
method, i.e. the main thread treats the completion of the blocking operation as
an event and register corresponding event handlers to handle this event.

The first technique is easy to use and easy to program. The program can be
written in traditonal way without relying on callbacks. However, the first
technique lacks efficiency, as launching too many threads compromises the
runtime performance. On the other hand, the second technique has superior
runtime performance, as it only maintains a single thread. However, it is
difficult to write programs using the second technique due to excessive use of
callbacks.

Lwt uses promises and cooperative threads to support non-blocking. The runtime
performance of Lwt is very good, as it only maintains a single physical thread
like the second technique. And it is quite easy to use. Writing asynchrounous,
fully non-blocking programs using Lwt is just like writing synchronous, blocking
programs using the first technique.

%\verb!sdfs\_dff!

%\begin{verbatim}
%Text enclosed inside environment
%is printed directly
%and all \LaTeX{} commands are ignored.
%\end{verbatim}

In the following sections, to facilitate understanding, I simplify some concepts
related with exception handling and modify the name of some important types in
Lwt.

\subsubsection{Promise State}

\begin{figure}[!h]
        \centering
        \includegraphics[width=1\columnwidth]{figure/promise-state.pdf}
        \caption{The state of a promise. The blue arrow represents how future
          states are transitioned. The red arrow represens whether can a
          programmer construct a certain state.}
        \label{fig:promise-state}
\end{figure}

The basic building block in Lwt is promise. A promise, as shown in figure
\ref{fig:promise-state}, has three states, which are \textbf{Sleep} state,
\textbf{Ready} state and \textbf{Link} state. The \textbf{Sleep} state
represents that the result of the promise is not available yet and one has to
wait before the \textbf{Sleep} state transitions into \textbf{Ready} state. The
\textbf{Sleep} state promise may contain a callback function, which is
immediately called after it is transitioned to \textbf{Ready}
state. \textbf{Ready} state represents that the result is available and we can
peek the result by checking the result field. Finally, the \textbf{Link} state
contains a pointer to a promise that is in \textbf{Sleep} state. It is used to
when waiting for multiple events.

The state of a proimse can be changed. The blue arrow of figure
\ref{fig:promise-state} shows the state transition graph between the three
states. Only \textbf{Sleep} state can generate a state transition.

When programming with promises, the programmer can only construct \textbf{Sleep}
state promise and \textbf{Ready} state promise. The \textbf{Link} state promise
is implicited constructed when chaining a promise with an annoynamous function
(we omit the discussion, as it does not affect the understanding of how promises
work).

\subsubsection{Chaining Promises}

\begin{verbbox}(>>=) : Promise a -> (a -> Promise b) -> Promise b \end{verbbox}

\begin{figure}[!h]
\resizebox{0.95\columnwidth}{!}{\theverbbox}
\caption{The infix operator for chaining promises.}
\label{fig:infix}
\end{figure}



Multiple promises can be chained together to accomplish complicated
tasks. Chaining is done through an infix operator \verb!>>=! (the then member
function of future in Seastar) as shown in figure \ref{fig:infix}, whose
signature is listed below.

Here, \verb!Promise a! represents a promise that, when transitioning into
\textbf{Ready} state, contains a result field with type \verb!a!. \verb!(a -> Promise b)!
represents an annouymous function, that takes a value of type \verb!a! as
argument and returns a value with type \verb!Promise b!. \verb!>>=! is a
function, that takes a value of \verb!Promise a! and an anouymous function of
\verb!(a -> Promise b)! and returns a value of \verb!Promise b!.

The real power of the \verb!>>=! operator is to chain multiple promises together
into a complicated operation. We give a piece of example code in figure
\ref{fig:example}. The code will first sleep for 3 seconds, then print "first
print" on the screen, then sleep another 3 seconds, and finally print "second
print" on the scrren. The best thing about this code is that, even if it
represents the consecutive execution of four blocking operations, but the code
itself is not blocking at all. It will return a promise in \textbf{Sleep} state
after being called. The blocking operations are implicitly handled by a
background thread.

\begin{verbbox}
  sleep 3 >>=
  fun () -> async_print "first print" >>=
  fun () -> sleep 3 >>=
  fun () -> async_print "second print"
\end{verbbox}

\begin{figure}[!h]
\resizebox{0.95\columnwidth}{!}{\theverbbox}
\caption{Chaining multiple promises into a complicated operation. The code above
  will first sleep for 3 seconds, then print "first print" on the screen, then
  sleep another 3 seconds, and finally print "second print" on the scrren.} 
\label{fig:example}
\end{figure}

\subsubsection{Annatomy of the infix operator}

\begin{verbbox}
(>>=) x f =
  match x with
  | Result r -> f r
  | Sleep ->
    let res = make_Sleep_promise () in
    add_callback x ( fun x -> connect res (x >>= f) )
    res
  | Link p ->
    assert false

connect t t' =
  match t' with
  | Result r -> (run t's callback)
  | Sleep ->
    (let t' link to t)
  
\end{verbbox}
\begin{figure}[!h]
\resizebox{0.95\columnwidth}{!}{\theverbbox}
\caption{The implementation of the infix operator. } %Depending on the state of the
  %promise \verb!x!, \verb!>>=! steps into three branches. If \verb!x! is a
 %result promise, then the annoynamous function.}
\label{fig:infix-internal}
\end{figure}

\noindent \textbf{The internal of the infix operator.} Depending on the state of
promise \verb!x!, the infix operator may step into two branches. If \verb!x! is
in \textbf{Result} state, then the anonymous function is immediately
excuted. But if \verb!x! is in \textbf{Sleep} state, the operator first creates a new
promise \verb!res! in \textbf{Sleep} state. Then a callback is added to \verb!x!. When
\verb!x! becomes \textbf{Result} state, the callback function is called, which
connects \verb!res! and \verb!x >> f!. The \verb!connect! function means that
the state of promise \verb!t'! will be reflected in promise \verb!t!.

\noindent \textbf{A conceptual explanation.} When executing \verb!x >>= f!, if
\verb!x! is immediately available and in \textbf{Result} state, then the
execution of the anonymous function \verb!f! continues without any
interruption.

The tricky part comes when \verb!x! is in \textbf{Sleep} state. It implies that
the result of \verb!x! is going to be available in the future. To prevent the
infix operator from blocking, we construct a new \textbf{Sleep} state promise
\verb!res! and returns \verb!res! to the user. Apparently, \verb!res! should
reflect the actual result of \verb!x >>= f!. To achieve this, before returning
\verb!res! to the user, we add a callback function to \verb!x!, which manually
connect \verb!res! with \verb!x >>= f! when \verb!x! becomes ready. In this way,
we mannualy construct a fully non-blocking abstraction.


\subsubsection{Wake Up Sleep Promise}

If a promise is in \textbf{Sleep} state, it must be waken up in the future and
transform into \textbf{Result} state. This is done by an external polling thread
which polls asynchronous completion messages. If the blocking operation that a
\textbf{Sleep} promise waits for has completed, the corresponding promise is
waken up.

\subsection{Seastar}

Seastar is basically a re-implementation of lwt in C++. It differes from lwt:

\textit{First}, the infix operator \verb!>>=! that chains promises together becomes
\verb!then! member function.

\textit{Second}, Seastar introduces a new object called \verb!future!, which is actually
a pointer object to the underlying promise. The reason is that C++ has no
garbage collection and Seastar has to manually manage the the memory allocation
of the promise object. Currently, Seastar either allocates the promise object
directly on the heap, or captures the promise object inside the callback
function. Therefore, to query the promise, seastar creates a pointer object
future, which contains a pointer to the underlying promise object.


\subsection{Promises in NFV}

We use promises to hide any kind of blocking operations when processing
packets. We give some usage examples.

\subsubsection{Simple Packet Processing}

\begin{verbbox}
xxx.then([]{
  process packet;
  return new_future;
}).then([]{
  process packet;
  return new_future;
})
  
\end{verbbox}
\begin{figure}[!h]
\resizebox{0.5\columnwidth}{!}{\theverbbox}
\caption{Example code of a simple packet processing software.} 
\label{fig:spps}
\end{figure}

It is straightforward to implement a simple packet processing software using
promises, as shown in figure~\ref{fig:spps}. The multiple chained promises
represent multiple processing stages on a packet processing pipeline.

\subsubsection{Using Promises to Hide Blocking During File IO}

\begin{verbbox}
xxx.then([]{
  do file IO;
  return new_future;
}).then([]{
  resume packet processing;
  return new_future;
})
  
\end{verbbox}
\begin{figure}[!h]
\resizebox{0.5\columnwidth}{!}{\theverbbox}
\caption{Example code of a NF that performs file IO.} 
\label{fig:file-io}
\end{figure}

Figure \ref{fig:file-io} represents a NF that performs file IO in the middle of
packet processing. Previously, this incurs kernel context switches and blocking,
which may compromise the performance of the NF. But with promise, the kernel
context switches and blocking are hidden by the promises, making the entire
operation fully non-blocking.

After issuing \verb!do file IO!, the NF software can continue to process other
flows. When file IO finishes, the processing of the original packet that incur
the file IO is resumed. This can greatly boost the performance of NFs that need
to perform file IO, such as PRADs.

\subsubsection{Handling Shared Variable Accessing}

\begin{verbbox}
xxx.then([]{
  access shared variable;
  return new_future;
}).then([]{
  resume packet processing;
  return new_future;
})
  
\end{verbbox}
\begin{figure}[!h]
\resizebox{0.5\columnwidth}{!}{\theverbbox}
\caption{Example code of a NF that needs to access shared variable.} 
\label{fig:asv}
\end{figure}

Figure \ref{fig:asv} shows an example of how to use promise to handle shared
variable processing. After issuing \verb!access shared variable!, the processing
is temporarily suspended and a request is created and sent to another thread to
access the shared variable. After modifying the shared variable, the suspended
packet processing is resumed.

By using promises, we can effectively linearize shared variable
accessing. Therefore we can keep the primary NF instance and the backup NF
instance in the same state, which simplifies primary-backup replication even in
multi-threaded environment.




%\bibliographystyle{abbrv}
%\bibliography{Bibliography}

%\end{document}

% TEMPLATE for Usenix papers, specifically to meet requirements of
%  USENIX '05
% originally a template for producing IEEE-format articles using LaTeX.
%   written by Matthew Ward, CS Department, Worcester Polytechnic Institute.
% adapted by David Beazley for his excellent SWIG paper in Proceedings,
%   Tcl 96
% turned into a smartass generic template by De Clarke, with thanks to
%   both the above pioneers
% use at your own risk.  Complaints to /dev/null.
% make it two column with no page numbering, default is 10 point

% Munged by Fred Douglis <douglis@research.att.com> 10/97 to separate
% the .sty file from the LaTeX source template, so that people can
% more easily include the .sty file into an existing document.  Also
% changed to more closely follow the style guidelines as represented
% by the Word sample file.

% Note that since 2010, USENIX does not require endnotes. If you want
% foot of page notes, don't include the endnotes package in the
% usepackage command, below.

% This version uses the latex2e styles, not the very ancient 2.09 stuff.
\documentclass[letterpaper,twocolumn,10pt]{article}
\usepackage{usenix,epsfig,endnotes}

\usepackage{graphicx, color}
\usepackage[font=small,labelfont=bf]{caption}
\usepackage[subrefformat=parens]{subcaption}
\usepackage{amsmath}
\usepackage{subcaption}
\usepackage{url}
\usepackage{listings}
\usepackage{verbatimbox}

\begin{document}

%make title bold and 14 pt font (Latex default is non-bold, 16 pt)
\title{\Large \bf NetStar: A Framework to Program Asynchronous Middleboxes}

\maketitle

\subsection*{Abstract}
Bla. Bla. Bla. Bla. Bla. Bla. Bla. Bla. Bla. Bla. Bla. Bla. Bla. Bla. Bla. Bla. Bla. Bla. Bla. Bla. Bla. Bla. Bla. Bla. Bla. Bla. Bla. Bla. Bla. Bla. Bla. Bla. Bla. Bla. Bla. Bla. Bla. Bla. Bla. Bla. Bla. Bla. Bla. Bla. Bla. Bla. Bla. Bla. Bla. Bla. Bla. Bla. Bla. Bla. Bla. Bla. Bla. Bla. Bla. Bla. Bla. Bla. Bla. Bla. Bla. Bla. Bla. Bla. Bla. Bla. Bla. Bla. Bla. Bla. Bla. Bla. Bla. Bla. Bla. Bla. Bla. Bla. Bla. Bla. Bla. Bla. Bla. Bla. Bla. Bla. Bla. Bla. Bla. Bla. Bla. Bla. Bla. Bla. Bla. Bla. Bla. Bla. Bla. Bla. Bla. Bla. Bla. Bla. Bla. Bla. Bla. Bla. Bla. Bla. Bla. Bla. Bla. Bla. Bla. Bla. Bla. Bla. Bla. Bla. Bla. Bla. Bla. Bla. Bla. Bla. Bla. Bla. Bla. Bla. Bla. Bla. Bla. Bla. Bla. Bla. Bla. Bla. Bla. Bla. Bla. Bla. Bla. Bla. Bla. Bla. Bla. Bla. Bla. Bla. Bla. Bla. Bla. Bla. Bla. Bla. Bla. Bla. Bla. Bla. Bla. Bla. Bla. Bla. Bla. Bla. Bla.

%\section{Introduction}

% nfv潮流试图将硬件middlebox替换为运行在虚拟环境中的软件middlebox,从而使关键话网络服务的部署和供给变得简单。由于软件middlebox需要被部署在关键的网络节点上,比如lte网络的骨干网络上,软件middlebox必须具有极高的性能。

The trend of Network Function Virtualization (NFV) \cite{nfv-website} aims to replace hardware middleboxes with software middleboxes running in virtualized environment. NFV greatly facilitates the deployment and provisioning of key network services. Since software middleboxes must be placed on critical network paths, such as backbone of LTE network \cite{201569}, the performance of software middleboxes must be good enough to process packets at 10/40Gbps line rate.

Besides being high performance, the ability to handle asynchronous operations is also very important to NFV. First, some middlboxes need to collaborate with each other by passing requests and responses. For instance, the middleboxes on the control plane of IMS system \cite{3gpp-ims} exchange a large number of SIP \cite{sip} requests and responses to establish a IP voice call. Second, middleboxes sometimes need to contact external service while processing flows. Stateless network function \cite{201545} stores critical flow states on an external key-value store \cite{ousterhout2015ramcloud}, for scalability and resilience. Middlebox that handle application-level protocols, like the middleboxes on the control plane of IMS system \cite{}, need to query a DNS server to identify the next-hop middlebox instance to contact with. To achieve good performance, all these middleboxes that are mentioned are preferably to handle asynchronous operations in a fully non-blocking manner.

% 为了提高软件middlebox的性能,主流的做法是跨越整个内核,并让软件middlebox直接在用户空间利用高效包处理库进行包处理。这种架构通常提供若干个工作线程直接对网卡进行繁忙轮训,以避免发送和接受包时由于内核用户空间的上下文切换而造成的巨大overhead,而这种overhead会对每秒钟需要处理几百万个包的软件middlebox产生极大的性能影响。

However, when implementing middleboxes with the state-of-art technique, non-blocking asynchronous operations can not be gracefully handled in a way that is both efficient and manageable. Today, most high-performance middleboxes bypass the kernel networking stack and process packets completely in user space with user space packet I/O frameworks, such as DPDK \cite{} and Netmap \cite{}. The user space packet I/O frameworks assign several worker threads to busily poll the network interface card (NIC) for packets, so that the overhead caused by context switches when calling traditional kernel I/O system calls is completely avoided. The use of a busy poll loop makes user-spacke packet I/O framework both efficient and easy to program with. But when incorporating asynchronous operations into the user-space packet I/O framework, a dilemma is encountered. If one would like the middlebox implementation to be easily manageable, he can replace asynchronous operations with synchronous blocking operations, but that will seriously damange the runtime performance of middleboxes, as high-performance middleboxes can not even tolerate the overhead brought by context switches. If one would like the middlebox implementation to be highly efficient, he can achieve the goal with a combination of mutable state and callback functions, but this makes middlebox implementation hard to manage, as a middlebox implementation may require to query a key-value store several times in order to process a single packet \cite{}.

%However, the use of user space I/O frameworks also limit the dominant working modes of software middleboxes to the following two, which are run-to-completion mode and pipelining mode. Run-to-completion \cite{} mode handles the fetching, processing and releasing of packets all within a single poll loop. Pipelining mode \cite{} distributes the packet processing job to multiple worker threads, each runs its own poll loop: when the poll loop of a worker thread finishes processing a packet, it hands the packet over to the next worker thread on the pipeline, or releases the packet to the outside if it is the last worker thread on the pipeline.

% 但是,这种基于用户层高性能网络报io的架构也限制了软件middlebox的工作模式,分别是run-to-completion模式和pipelining mode。run-to-completion mode在一个poll loop中完成从抓包,处理包,到把包释放出的全部过程。pipelineing mode则把包处理工作分配到多个工作线程中,当一个工作线程的poll loop完成自己的任务后,它会将包交给下一个工作线程,直到这个包被pipeline上的最后一个工作线程释放出去。这两种工作方式十分高效,并且容易编程实现。但是,这两种工作方式通常只能使用同步的方式来响应异步事件,例如进行dns查询并等待它的返回结果。这种同步的方式会极大的影响middlebox的性能。



%Both of the two working modes are highly efficient and easy to program with. However, it's hard for these two working modes to handle non-blocking asynchronous operations, i.e. perform a DNS query and wait for the response. It is trivial to integrate a blocking synchronous operation into the poll loop, but blocking synchronous operations are not efficient and will seriously damage the runtime performance of software middleboxes. It is possible to use callbacks to perform asynchronous operations, but this makes the middlebox code hard to manage, as compared to synchronous operations.



% 现有的nfv系统中也有很多响应异步事件的,例如mOS。mOS将poll loop进行了分装,并抽象出包处理过程中的事件,这些事件会被用户注册的回调函数进行响应,从而实现异步的事件处理。但是,当程序员使用mOS时,他需要使用回调函数来串接处理逻辑。和传统的工作模式相比,这种方法会打乱程序的控制流,从而使得程序员更难reason about程序。同时,mOS所暴露出的事件使基于包到达以及tcp状态机的改变的事件,这些事件并不general,无法捕捉更多工作,例如查询外部的存储设备。

There are several existing frameworks that aim to provide asynchronous event processing for middleboxes, including mOS \cite{}, libnids \cite{} and etc. These frameworks expose TCP related flow events to the middlebox programmer and enable these events to be processed in a non-blocking asynchronous manner. However, these frameworks are still based on callback-based design, making the code harder to manage when handling complex middlebox logic. They only concentrate on handling flow-level events that are related to TCP/IP protocol and they are not general enough to handle other asynchronous operations like database query.

% 在这篇论文中,我们呈现NetStar,一个为middlebox的编写提供highly scalable, lightweight异步事件处理的编程平台。NetStar基于开源事件驱动框架Seastar进行构建并使用了先进的promise-continuation编程模型。与之前的系统比较,NetStar使得middlebox可以灵活的,无阻塞的响应各种异步事件,同时也使的异步的middlebox更容易去reason,更容易实现。我们必须承认的是,为了实现高效异步事件处理,NetStar确实使用了一些抽象,但是这些抽象只会带来moderate overhead,而且NetStar仍然具有极强的多核扩展能力。

In this paper, we present NetStar, a framework for implementing middleboxes that perform non-blocking asynchronous operations. NetStar is built upon open-source asynchronous library Seastar \cite{} and provides basic constructs for building middleboxes. When compared with previous frameworks, NetStar exhibit the following advantages.

\textbf{NetStar handles asynchronous operations of middleboxes in a way that is both efficient and manageable.} Asynchronous operations in NetStar are accomplished through through callbacks, making NetStar highly efficient. However, the callbacks in NetStar are used in an implicit way that mimics the style of synchronous operations, making them easy to program with and reason about. NetStar's power comes from the promise-continuation programming model provided by Seastar \cite{} and advanced C++ features, such as lambda expression \cite{}.

\textbf{NetStar has good multi-core scalability and performance.} NetStar achieves multi-core scalability with hardware assistance. The input network traffic are automatically distributed to each core by NIC using RSS \cite{}, so that so that different cores can operate in a fully parallel fashion without contending for shared resources. NetStar is programmed with C++14 using various zero-cost abstractions. Even though the promise-continuation model that NetStar uses to implement asynchronous operations do pose slight runtime overhead, NetStar is able to compensate the overhead with multi-core scalability and achieve line-rate processing.

Using NetStar, it would be easy to implement traditional asynchronous operations such as DNS querying. Instead, we use NetStar to solve some important research problems that are raised recently.

\textbf{First,} we use NetStar to implement a non-blocking version of stateless network function \cite{}. Stateless network function \cite{} has demonstrated great potential for dynamic scaling and resilience. However, its current implementation is blocking and synchronous, making its maximum performance inferior when compared to other state-of-art middlebox systems \cite{}. We use NetStar to transform stateless network function into a non-blocking and asynchronous one and pairs it with a key-value store \cite{} with better performance than RamCloud \cite{}. Besides significantly improving the performance of stateless network function, we further identify a race problem when accessing the shared state stored on a key-value store by stateless network function. Without using NetStar, it would be hard to solve such a problem.



% 利用NetStar,我们可以很轻松的实现异步DNS查询等传统功能。为了展现NetStar的真正能力,我们着重解决最近研究界提出的几个问题:

% 实现异步无阻塞的Stateless Network Function: Stateless Network Function架构对于网络功能的容错和动态扩展都有重要价值。但是,现有的Stateless Network Function是基于一个完全同步的架构,工作线程需要对异步操作进行阻塞性等待。这使得Stateless Network Function无法有效的利用CPU资源。我们利用NetStar对Stateless Network Function进行了重写,将所有的同步等待转换成异步操作,从而极大的提高了Stateless Network Function的运行效率,同时保持了Stateless NF代码逻辑的可读性。我们也发现了Stateless Network Function在处理共享状态时的一个问题,并利用NetStar框架对其进行了有效的解决。

% 共享数据结构的访问:为了序列化共享数据结构的访问,传统NF需要对共享数据结构进行加锁。我们则利用NetSeatar框架,将访问共享数据结构的过程做成了一个消息传递过程:共享数据结构被做成了一个服务,所有工作线程对它的访问都被转换成了发送消息并接收消息访问结果。这样做保证了共享数据结构访问的线性化,同时,我们在实验中展示这样做的性能损耗并不比利用锁来处理多多少。但是这样做可以带来两个好处,首先,这样做可以解决上述提到的Stateless Network Function在处理共享状态时的多线程race问题,第二,我们可以有效记录共享数据结构的访问顺序,并有效的按照锁记录的访问顺序复制对共享数据的访问。这使得我们可以构建一个主机从机完全一致的主从复制系统,我们在实验中展示了这个系统的性能。

\textbf{Second,} we use NetStar to serialize access to shared data structures of middleboxes. Traditional middleboxes need to use lock to serialize the access to a shared data structures from multiple worker threads. We treat shared data structures as a dedicated service and convert the regular access-after-lock pattern to send-query-wait-response pattern. The new pattern is reasonably fast when compared to the access-after-lock pattern. It provides two additional benefits. First, it helps us solve the shared state access problem of stateless network function mentioned above. Second, it enables the middlebox to record the access order of any shared variables, so that another middlebox can replay the recorded access order in a deterministic fashion. We use this feature to build a primary-backup replication system for middleboxes. It's primary and back instances are always in synchronization and it has zero recovery time after primary failure.

% 作为一个探索异步操作在middlebox中实现的平台,NetStar并没有去port很多已有的middlebox。坦白来讲,由于NetStar使用了新的编程抽象,并且大量使用了C++14的语法特性,将已有的middlebox port到NetStar需要一定additional的工作量。但是,本文的重点是为了展示NetStar对支持middlebox异步操作的有效性,我们所制作的全异步Stateless Network Function相比同步的版本有重大的性能提升。其次,我们相信NetStar有足够的能力去支持更多的middlebox,NetStar基于的SeaStar平台带有内建的TCP/IP stack,这让我们有充足的理由去相信,NetStar可以支持支持mOS,因为mOS基于mTCP。

Due to the use of new programming model and massive use of C++14 feature, porting existing middleboxes to NetStar may require non-trivial amount of work. However, as a new framework for exploring how asynchronous operations can be implemented in middleboxes, porting existing middleboxes is not our primary goal. We also believe that the constructs provided by NetStar for building middleboxes are general enough to implement other middleboxes that do not perform asynchronous operations.

The paper is organized as follows. We introduce our primary motivation in section 2. The overall architecture of NetStar is given in section 3. We discuss the design of asynchronous stateless network function and serialized shared data structure access in section 4. The performance of asynchronous stateless network function and a NAT is discussed in section 5. We discuss related work in section 6 and conclude the paper in section 7.

%这篇文章的组织如下。我们在第二章介绍我们的主要motivation。在第三张,我们对NetStar框架进行一个详细介绍。在第四章,我们详细介绍异步stateless middlebox和共享数据结构的访问。我们在第五章呈现我们的evaluation结果。在第六章讨论related work。并在第七章给出论文的结论。

%\section{Background and Motivation}

In recent years, the research community has witnessed the quick development of
network function virtualization (NFV). DPDK \cite{dpdk} and Netmap
\cite{rizzo2012netmap} use kernel bypassing to speed up the performance of NF
software. They have become the default libraries for implementing high-speed
modern NF software. NFV management systems such as E2 \cite{palkar2015e2} are
built to dynamically scale virtual instances running different NFs. NFs are
augmented with fault tolerance \cite{sherry2015rollback} and flow migration
\cite{gember2014opennf} to improve the failure resilience.

However, despite all these advancements, a core problem is not well-solved by
existing work: what should be the default programming abstraction for implementing NF
software, so that the diverse requirements of NF software can be well-captured
by this abstraction? To show the importance of this problem, let me first discuss
the diversity of NF software.

\subsection{Diversity of NF Software}


\noindent \textbf{Simple Packet Processing Program.} Example NFs include
firewall, NAT and load balancer. The word ``simple'' actually means that the way
that these NFs manipulate packets is simple: they take an input packet,
perform necessary packet transformation and book-keeping, then they release the
packet to the outside. Taking NAT as an example. After receiving an input
packet, NAT may update the connection status associated with the flow, then the
NAT performs an address translation to substitute the IP address and port of the
packet. Finally, NAT sends the packet out from the output port.

\textit{These NFs can be effectively implemented inside a polling loop and can be
seamlessly integrated with either DPDK or Netmap for maximum performance.}

\noindent \textbf{NFs with Intensive File I/O.} Example NFs include PRADs
\cite{prads} asset monitor and Snort \cite{snort} intrusion detection system
(IDS). For instance, PRADS is a passive real-time asset detection system, which
listens to network traffic and logs important information on hosts and services
it sees on the network. This information can be used to map the underlying
network, letting network operators know what services and hosts are active, and
can be used together with IDS/IPS setup for "event to application" correlation.

Both PRADs and Snort can be ported to use DPDK to speed up packet processing
\cite{201546}. Even after porting to DPDK, both NFs fail to achieve 10Gbps line
rate processing \cite{201546}. The primary reason for this undesirable number is
due to logging. After porting to DPDK, the worker threads of both NFs keep
polling for new packets and maintain CPU usage to 100\%. But when both NFs log
important events, they have to access system calls related to file system
processing, generating expensive context switches and compromising the packet
processing throughput.

\textit{These NFs can be accelerated using DPDK and Netmap, but they still need
  to step into the kernel to log events to the files.} NFs with intensive file
I/O remain to be interesting phenomena in existing NFV research. People have
strived to remove context switches associated with kernel networking stack by
bypassing the kernel with DPDK, but they fail to remove the context switches
associated with kernel file systems during logging.

\noindent \textbf{NFs with Reliable Communication to External Services.}
Example NFs include S-CSCF in IMS system \cite{3gpp-ims} and NFs that need to
replicate their states on back NFs.

S-CSCF is an important middlebox sitting at control plane of the IMS system. It
processes SIP \cite{sip} messages by contacting several external
services. Taking the S-CSCF implementation of a famous open source IMS project
Clearwater \cite{project-clearwater} as an example, when processing SIP messages,
S-CSCF needs to log SIP registration information on a Memcached \cite{memcached}
cluster and acquire user information by querying a dedicated storage server
called Home Subscriber Server (HSS). The S-CSCF implementation of Clearwater
uses kernel TCP/IP stack to carry out reliable communication to all the required
external services, seriously limiting the maximum throughput that S-CSCF can
achieve. Our experience with Clearwater shows that a single worker thread in
S-CSCF can only process SIP messages with the bandwidth of 40Mb.

FTMB \cite{sherry2015rollback} is the state of art system for NF replication. It
employs a primary-backup replication strategy. On the primary NF instance,
after each packet is processed, the packet is passed to the backup over a
reliable communication channel for replication. We can treat the replication
process as communicating external services: each input packet processed by the
primary instance must be reliably delivered to the backup instance. FTMB uses
DPDK to speed up packet processing and implements its own reliable communication
channel on top of DPDK. But the implementation detail of the reliable
communication channel is omitted from the paper. It would be desirable to
implement the reliable communication channel using a user-level TCP/IP stack
like mTCP \cite{179773}, so that the performance of FTMB is stable (a
handcrafted reliable communication channel may be unstable and lack of flow
control) and it is easier to reproduce FTMB implementation for both academic and
industrial usage. \textit{However, without a good programming abstraction,
  integrating a user-level TCP implementation like mTCP with replication
  strategy like FTMB is not a trivial task:} mTCP exposes an event-driven
programming interface like Linux epoll. The application thread using mTCP does
not sit in the same thread as the mTCP worker thread. But FTMB requires that the
same worker thread handles both NF packet processing and reliable communication
to ensure correct replication.

\textit{Some of these NFs abandoned DPDK
  and Netmap, use kernel networking stack to provide reliable communication
  channel, but sacrifice performance. Some of these NFs use DPDK and Netmap to
  speed up packet processing and implement their own reliable communication
  channel, but sacrifice the stable performance and flow control provided by TCP/IP. }



\noindent \textbf{NFs that Process Events Raised by Lower-level System
  Components.} Example NFs include Snort IDS \cite{snort} and Bro IDS
\cite{bro}. The two IDSes alert potential attacks by matching the flow protocols
and analyzing flow payloads with an automaton. They can be decoupled into two
parts: A low-level system is responsible for re-assembling the TCP stream and
generating events associated with the TCP stream, i.e. connection setup,
packet re-transmission, and the new packet payload. A high-level event driven
system is responsible for reacting to the events raised by the low-level system,
i.e. in the case of a fake re-transmission forged by an attacker, the IDS drops the
flow and raises an alert. These IDSes can be effectively accelerated using mOS
\cite{201546}, which substitute the low-level system that raises flow-related
events. mOS is accelerated using DPDK and is an improved version of mTCP \cite{179773}.

\textit{The low-level system of these NFs can be accelerated with DPDK. However,
  the low-level system like mOS is usually targeted to process TCP/IP protocol
  and can not be extended to process non-TCP/IP protocol.}

\noindent \textbf{Summary.} Now we briefly discuss the similarities and
differences of all the discussed NF software.

\noindent \textbf{Similarity.} Most of these NFs can be accelerated with DPDK or
Netmap (except for S-CSCF, which relies on kernel networking stack, but we can
still accelerate it by porting it to user-level TCP/IP stack like mTCP). Using
DPDK or Netmap means that the worker threads in these NFs become busy polling
thread that keeps CPU usage to 100\%, implying that any system calls entering
the kernel context may compromise the performance of these NFs.

\noindent \textbf{Difference.} These NFs have different working goals and
operate at different levels. Simple packet processing programs only manipulate
raw packets. They do not rely on any external services. PRADs needs to do file
I/O. FTMB and S-CSCF need to communicate with external services. Snort and Bro
operate on a high-level that reacts to flow-level events raised by a low-level
system components. These differences lead to diverse implementation details,
making it hard to find an appropriate abstraction to unify these NFs. 

\subsection{One Abstraction to Rule Them All}

Just like the dedication that physicists put into the grand unified theory,
computer scientists also have been searching for a unified programming
abstraction that can capture a variety of applications. In terms of NFV, if a
unified programming abstraction can be found for all the NFs mentioned in
the previous section, programmers can enjoy the following benefits.

First, by optimizing the performance of the library that provides the
unified programming abstraction, we can improve the performance for a huge variety of
NFs. There is no need to optimize each NF, which might take a huge amount of
labor work.

Second, ease NF software development. Once the implementor becomes familiar with
the programming abstraction, he is able to create different types of NFs without
learning different programming paradigms or constructing different libraries.

Finally, it makes important research and industrial result easily reproducible,
as the unified programming abstraction makes people play on the same ground.

Such a programming abstraction is readily accessible for NFV implementors and
researchers, which is functional reactive programming, especially the subset
related to futures, promises and continuations.

\subsection{Futures and Promises}

Futures and promises are important terminologies in functional reactive
programming. A future represents a value that is going to be computed while a
promise represents the action when the computation is done. This simple
programming paradigm can easily capture most of the asynchronous programming
patterns. Let me briefly explain how futures and promises can be mapped to NFs
discussed in previous sections.

For simple packet processing program, the futures are packets that are going to
be received whereas the promises are packet handler functions.

For PRADs, the futures and promises can be combined to implement efficient file
system logging. The futures represent the logging action that will log events
raised by PRADs to the file system. The promises represents post actions when
the logging is done.

For S-CSCF and FTMB, futures and promises can be used to implement an efficient
user-space TCP/IP stack. The futures are still packets to be received, but the
promises become TCP/IP stack handlers.

For Snort and Bro, futures and promises can be used to implement a low-level
system that raises flow events. Futures flow events that are going to be raised,
promises are event handlers for these events.

The future-promise programming abstraction can be efficiently implemented with
a small runtime overhead (i.e. asynchronous C++ library Seastar
\cite{seastar}). The programming abstraction can fully bypass the entire kernel,
even in terms of file logging (with the help of DMA), providing satisfactory
performance for modern NF software.

\subsection{Contribution}

In this paper, we are going to make the following contributions.

First, we are the first to apply functional reactive programming as a generic
method for building a variety of NF software. We use seastar as the underlying
library for providing the reactive programming abstraction.

Second, we carry out case studies to show how functional reactive programming
can be used to construct 4 different types of NFs, with diverse requirements.

Finally, we show that the performance and ease of implementation are greatly
improved by using functional reactive programming. In particular:

\noindent \textbf{We re-implement PRADs using functional reactive programming.}
The resulting PRADs is capable of logging to file system at a throughput of
several gigabits per second.

\noindent \textbf{We create a new primary-backup replication strategy.} The new
primary-backup strategy is capable of processing packets at line rate. The
biggest difference between this replication strategy with FTMB is that it does
not need to checkpoint the master NF instance, greatly simplifying the
implementation effort. It has no replay time and introduces no extra latency
caused by checkpointing.

\noindent \textbf{We re-implement mOS using our new programming abstraction.} We
also port PRADs to use the new mOS and show that the new PRADs can be several
times faster than that in the mOS paper \cite{201546}.


%\section{Primary-backup NF Replication Without Rolling Back}
%To be continued.

%\input{section/second.tex}
%\section{Middlebox Recovery without Rolling Back}

On contrary to FTMB, we aim to provide a new middlebox replication strategy
which keeps both the primary and the backup instances in synchronization. This
replication strategy directly gives the same input packet stream to both the
primary instance and the backup instance, our new architecture is able to keep
both of the two instances in synchronization.

Using this strategy, there is no need to roll-back the backup instance when the
primary instance fails. This strategy minimizes the recovery time and eliminates
the prolonged packet processing delay when checkpointing the primary instance.

To tackle the challenge of keeping both the primary instance and the backup
instance in synchronization, one might resolve to deterministic scheduling
\cite{}. However, the overhead caused by deterministic scheduling is way too
high for NF software, as a typical NF software needs to process millions of
input packets every second. Instead, we solve the deterministic execution
problem using a combination of coroutine and message passing.

\begin{verbbox} Temporarily removed. \end{verbbox}

\begin{figure}[!t]
  \begin{subfigure}[t]{0.5\linewidth}
    \centering
    %\includegraphics[width=\columnwidth]{section/overall.jpg}
    \resizebox{0.95\columnwidth}{!}{\theverbbox}
    \caption{The basic setup.}\label{fig:overall}
  \end{subfigure}\hfill
  \begin{subfigure}[t]{0.5\linewidth}
    \centering
    %\includegraphics[width=\columnwidth]{section/primary.jpg}
    \resizebox{0.95\columnwidth}{!}{\theverbbox}
    \caption{The execution flow on a primary instance.}\label{fig:primary}
  \end{subfigure}\hfill
  \begin{subfigure}[t]{0.5\linewidth}
    \centering
    %\includegraphics[width=\columnwidth]{section/backup.jpg}
    \resizebox{0.95\columnwidth}{!}{\theverbbox}
    \caption{The execution flow on a backup instance.}\label{fig:backup}
  \end{subfigure}
%Removed for better illustration.
\caption{Work flow of how to keep both primary and backup in synchronization.}
\label{fig:fig}
\end{figure}

\subsection{Basic Setup for Recovery without Rolling Back}

The basic setup is shown in figure \ref{fig:overall}. We set up two identical NF
instances. The two instances run the same NF software binary image and are
configured with the same number of CPU cores.

In \ref{fig:overall}, the NF instance is configured with two worker threads
($t_1$ and $t_2$) which poll the NIC card for input packet. The shared variable
is hosted on a dedicated thread $t_3$.

To access the shared variable hosted on $t_3$, both $t_1$ and $t_2$ need to send
a message for accessing the shared variable to $t_3$. After the shared variable
is modified, another message is sent back to $t_1$ or $t_2$ to indicate the
completion of shared variable modification.

When the primary instance finishes processing the input packet, it forwards the
input packet to the backup instance over a reliable communication channel. The
backup instance processes the input packet again before releasing the
packet. The packet processed by the backup instance is tagged with an execution
order by the primary, so that the shared variable on the backup instance can
process the input packet in the same sequence as the primary instance. This
guarantees that the state of both the primary instance and the backup instance
are always synchronized.

\subsection{Workflow on Primary Instance}

Figure \ref{fig:primary} shows how worker thread $t_2$ processes an input
packet. The overall workflow is similar to a typical packet polling loop. The
only exception is that when a shared variable is going to be accessed by $t_2$
(step 3 in figure \ref{fig:primary}), instead of directly acquiring the lock and
update the shared variable, $t_2$ sends the packet to $t_3$ to update the shared
variabale (step 4). When $t_3$ receives this packet, the threads update the
shared variable (step 5), tags the packet with a sequence number (step 6) and
sends the packet back to $t_2$. When $t_2$ receives this packet, $t_2$ sends the
tagged packet out to the backup instance over a reliable communication channel.

\subsubsection{The Sequence Number}

The sequence number tagged by $t_3$ indicates a sequential accessing order to
the shared variable. Using this sequence number, the backup instance can
reliably reproduce the accessing order of the shared variable (to be discussed
in section \ref{sec:backup}). This ensures that the state of the primary and backup
instances are always synchronized.

\subsubsection{Using Future and Promise to Cancel Thread Blocking}

The biggest problem with this workflow is that, after step 4 in figure
\ref{fig:primary}, worker thread $t_2$ must block its execution and wait for the
packet to come back from $t_3$. This is unacceptable for a high-performance NF
software.

We tackle this problem using futures and promises in reactive
programming. After step 4 in figure \ref{fig:primary} is executed, we wrap the
current thread context inside a future object. $t_2$ can immediately start processing
other input packets. When the packet comes back to $t_2$ after step
6, $t_2$ is able to re-construct the previous thread context using the
corresponding future object. This efficiently eliminates thread blocking.

\subsubsection{The Reliable Communication Channel}

Due to the power of future and promise, a user-space TCP/IP stack could be
integrated inside $t_2$. The reliable communication channel is actually a TCP
connection channel. This reliable communication channel is augmented with flow
control and is more reliable than other specially-crafted reliable communication protocols. 

\subsection {Workflow on Backup Instance}
\label{sec:backup}

Figure \ref{fig:backup} shows how worker thread $t_2$ processes the output
packet sent from the primary instance. The overall workflow is similar to that
of the primary instance. However, when the packet is delivered to $t_3$ to
access the shared variable, $t_3$ must check whether it has processed all the
packets whose sequence number is smaller than the packet. Considering the case
of figure \ref{fig:backup}, $t_3$ must wait for packet with sequence number 0
first. If $t_3$ has processed packet with sequence number 0, $t_3$ can directly
process the packet with sequence number 1. Otherwise, $t_3$ should store packet with
sequnece number 1 and wait for the packet with sequenece number 0 to come.

Since the order of how packets access the shared variable is well preserved,
the backup instance and the primary instance have the same state.

\subsection{Recovery}

If the primary instance fails, the backup instance can become the primary
instance immediately. The recovery time is basically decreased to zero.






%%\section{Asynchronous Programming}
%The core idea of asynchronous programming is to handle all the tasks that need
%to block waiting for results asynchronously, so that nothing is really blocked.

%It can be implemented using an active poll loop. The active poll loop polls
%different event source for events. Whenever an event happens, the poll loop
%executes a callback function associated with the event to handle the event.

%An improved asynchronous programming style is to use futures and promises. The
%future and promise abstraction can handle asynchronous programming in a nice
%manner.

%Asynchronous programming is the solution for all the problems that I have
%mentioned in the previous section.

%\section{Seastar and OSv}

%Seastar is a modern asynchronous programming library with future and promise
%style. It can be combined together with DPDK to achieve low-level packets
%processing. It also includes a user-space TCP/IP stack that is driven by
%asynchronous programming.

%It is the best asynchronous programming library that we can use to build network
%functions.

%Seastar can also run in OSv, a unikernel. In this way we can achieve
%high-performance virtualization.

%\section{Planned Roadmap}

%I plan to do the following case studies using Seastar in this work.

%First, implementing an unified NFV platform like NetBricks. Highlight that with
%the help of C++ unique\_ptrs, we can achieve the similar software memory safety
%like NetBricks.

%Second, port PRADS to use Seastar. Show that with the help of asynchronous
%programming, we can greatly improve the performance of PRADS.

%Third, implement a simplified version of mOS. Show that asynchronous programming
%can greatly simplify flow event generation and processing.

%Fourth, re-implement FTMB. Show that how easy it is to implement primary-backup
%replication for network functions using seastar.

%Fifth, a simplified SDN switch using Seastar, show it's performance boost when
%compared against OpenVSwitch.

\section{Promises and Cooperative Threads}

In this section, I will first give a detailed overview about promises and
cooperative threads used by Seastar fraemwork. Then I will introduce how to
apply promises and coopeartive threads to NFV.

\subsection{Lwt}

The core building blocks of Seastar are promises and cooperative
threads. Clearly, these fancy concepts come from the world of functional
programming languages. There is an Ocaml library called Lwt \cite{vouillon2008lwt}, which also
implements promises and cooperative threads. In the following sections, I will
first discuss how promises and coopeartive threads are implemented in
Ocaml. Then I will show how they are implemented in Seastar.

\subsubsection{Lwt Overview}

When wring programs like network servers, non-blocking is a very important
property that ensures the runtime efficiency of the network servers. There are
two dominant techniques to make the program non-blocking. The first one uses
multi-threading to support non-blocking, i.e. whenever a blocking operation is
going to be made, the main thread of the program lanuches another thread to
handle the blocking operation. The second one uses event-based programming
method, i.e. the main thread treats the completion of the blocking operation as
an event and register corresponding event handlers to handle this event.

The first technique is easy to use and easy to program. The program can be
written in traditonal way without relying on callbacks. However, the first
technique lacks efficiency, as launching too many threads compromises the
runtime performance. On the other hand, the second technique has superior
runtime performance, as it only maintains a single thread. However, it is
difficult to write programs using the second technique due to excessive use of
callbacks.

Lwt uses promises and cooperative threads to support non-blocking. The runtime
performance of Lwt is very good, as it only maintains a single physical thread
like the second technique. And it is quite easy to use. Writing asynchrounous,
fully non-blocking programs using Lwt is just like writing synchronous, blocking
programs using the first technique.

%\verb!sdfs\_dff!

%\begin{verbatim}
%Text enclosed inside environment
%is printed directly
%and all \LaTeX{} commands are ignored.
%\end{verbatim}

In the following sections, to facilitate understanding, I simplify some concepts
related with exception handling and modify the name of some important types in
Lwt.

\subsubsection{Promise State}

\begin{figure}[!h]
        \centering
        \includegraphics[width=1\columnwidth]{figure/promise-state.pdf}
        \caption{The state of a promise. The blue arrow represents how future
          states are transitioned. The red arrow represens whether can a
          programmer construct a certain state.}
        \label{fig:promise-state}
\end{figure}

The basic building block in Lwt is promise. A promise, as shown in figure
\ref{fig:promise-state}, has three states, which are \textbf{Sleep} state,
\textbf{Ready} state and \textbf{Link} state. The \textbf{Sleep} state
represents that the result of the promise is not available yet and one has to
wait before the \textbf{Sleep} state transitions into \textbf{Ready} state. The
\textbf{Sleep} state promise may contain a callback function, which is
immediately called after it is transitioned to \textbf{Ready}
state. \textbf{Ready} state represents that the result is available and we can
peek the result by checking the result field. Finally, the \textbf{Link} state
contains a pointer to a promise that is in \textbf{Sleep} state. It is used to
when waiting for multiple events.

The state of a proimse can be changed. The blue arrow of figure
\ref{fig:promise-state} shows the state transition graph between the three
states. Only \textbf{Sleep} state can generate a state transition.

When programming with promises, the programmer can only construct \textbf{Sleep}
state promise and \textbf{Ready} state promise. The \textbf{Link} state promise
is implicited constructed when chaining a promise with an annoynamous function
(we omit the discussion, as it does not affect the understanding of how promises
work).

\subsubsection{Chaining Promises}

\begin{verbbox}(>>=) : Promise a -> (a -> Promise b) -> Promise b \end{verbbox}

\begin{figure}[!h]
\resizebox{0.95\columnwidth}{!}{\theverbbox}
\caption{The infix operator for chaining promises.}
\label{fig:infix}
\end{figure}



Multiple promises can be chained together to accomplish complicated
tasks. Chaining is done through an infix operator \verb!>>=! (the then member
function of future in Seastar) as shown in figure \ref{fig:infix}, whose
signature is listed below.

Here, \verb!Promise a! represents a promise that, when transitioning into
\textbf{Ready} state, contains a result field with type \verb!a!. \verb!(a -> Promise b)!
represents an annouymous function, that takes a value of type \verb!a! as
argument and returns a value with type \verb!Promise b!. \verb!>>=! is a
function, that takes a value of \verb!Promise a! and an anouymous function of
\verb!(a -> Promise b)! and returns a value of \verb!Promise b!.

The real power of the \verb!>>=! operator is to chain multiple promises together
into a complicated operation. We give a piece of example code in figure
\ref{fig:example}. The code will first sleep for 3 seconds, then print "first
print" on the screen, then sleep another 3 seconds, and finally print "second
print" on the scrren. The best thing about this code is that, even if it
represents the consecutive execution of four blocking operations, but the code
itself is not blocking at all. It will return a promise in \textbf{Sleep} state
after being called. The blocking operations are implicitly handled by a
background thread.

\begin{verbbox}
  sleep 3 >>=
  fun () -> async_print "first print" >>=
  fun () -> sleep 3 >>=
  fun () -> async_print "second print"
\end{verbbox}

\begin{figure}[!h]
\resizebox{0.95\columnwidth}{!}{\theverbbox}
\caption{Chaining multiple promises into a complicated operation. The code above
  will first sleep for 3 seconds, then print "first print" on the screen, then
  sleep another 3 seconds, and finally print "second print" on the scrren.} 
\label{fig:example}
\end{figure}

\subsubsection{Annatomy of the infix operator}

\begin{verbbox}
(>>=) x f =
  match x with
  | Result r -> f r
  | Sleep ->
    let res = make_Sleep_promise () in
    add_callback x ( fun x -> connect res (x >>= f) )
    res
  | Link p ->
    assert false

connect t t' =
  match t' with
  | Result r -> (run t's callback)
  | Sleep ->
    (let t' link to t)
  
\end{verbbox}
\begin{figure}[!h]
\resizebox{0.95\columnwidth}{!}{\theverbbox}
\caption{The implementation of the infix operator. } %Depending on the state of the
  %promise \verb!x!, \verb!>>=! steps into three branches. If \verb!x! is a
 %result promise, then the annoynamous function.}
\label{fig:infix-internal}
\end{figure}

\noindent \textbf{The internal of the infix operator.} Depending on the state of
promise \verb!x!, the infix operator may step into two branches. If \verb!x! is
in \textbf{Result} state, then the anonymous function is immediately
excuted. But if \verb!x! is in \textbf{Sleep} state, the operator first creates a new
promise \verb!res! in \textbf{Sleep} state. Then a callback is added to \verb!x!. When
\verb!x! becomes \textbf{Result} state, the callback function is called, which
connects \verb!res! and \verb!x >> f!. The \verb!connect! function means that
the state of promise \verb!t'! will be reflected in promise \verb!t!.

\noindent \textbf{A conceptual explanation.} When executing \verb!x >>= f!, if
\verb!x! is immediately available and in \textbf{Result} state, then the
execution of the anonymous function \verb!f! continues without any
interruption.

The tricky part comes when \verb!x! is in \textbf{Sleep} state. It implies that
the result of \verb!x! is going to be available in the future. To prevent the
infix operator from blocking, we construct a new \textbf{Sleep} state promise
\verb!res! and returns \verb!res! to the user. Apparently, \verb!res! should
reflect the actual result of \verb!x >>= f!. To achieve this, before returning
\verb!res! to the user, we add a callback function to \verb!x!, which manually
connect \verb!res! with \verb!x >>= f! when \verb!x! becomes ready. In this way,
we mannualy construct a fully non-blocking abstraction.


\subsubsection{Wake Up Sleep Promise}

If a promise is in \textbf{Sleep} state, it must be waken up in the future and
transform into \textbf{Result} state. This is done by an external polling thread
which polls asynchronous completion messages. If the blocking operation that a
\textbf{Sleep} promise waits for has completed, the corresponding promise is
waken up.

\subsection{Seastar}

Seastar is basically a re-implementation of lwt in C++. It differes from lwt:

\textit{First}, the infix operator \verb!>>=! that chains promises together becomes
\verb!then! member function.

\textit{Second}, Seastar introduces a new object called \verb!future!, which is actually
a pointer object to the underlying promise. The reason is that C++ has no
garbage collection and Seastar has to manually manage the the memory allocation
of the promise object. Currently, Seastar either allocates the promise object
directly on the heap, or captures the promise object inside the callback
function. Therefore, to query the promise, seastar creates a pointer object
future, which contains a pointer to the underlying promise object.


\subsection{Promises in NFV}

We use promises to hide any kind of blocking operations when processing
packets. We give some usage examples.

\subsubsection{Simple Packet Processing}

\begin{verbbox}
xxx.then([]{
  process packet;
  return new_future;
}).then([]{
  process packet;
  return new_future;
})
  
\end{verbbox}
\begin{figure}[!h]
\resizebox{0.5\columnwidth}{!}{\theverbbox}
\caption{Example code of a simple packet processing software.} 
\label{fig:spps}
\end{figure}

It is straightforward to implement a simple packet processing software using
promises, as shown in figure~\ref{fig:spps}. The multiple chained promises
represent multiple processing stages on a packet processing pipeline.

\subsubsection{Using Promises to Hide Blocking During File IO}

\begin{verbbox}
xxx.then([]{
  do file IO;
  return new_future;
}).then([]{
  resume packet processing;
  return new_future;
})
  
\end{verbbox}
\begin{figure}[!h]
\resizebox{0.5\columnwidth}{!}{\theverbbox}
\caption{Example code of a NF that performs file IO.} 
\label{fig:file-io}
\end{figure}

Figure \ref{fig:file-io} represents a NF that performs file IO in the middle of
packet processing. Previously, this incurs kernel context switches and blocking,
which may compromise the performance of the NF. But with promise, the kernel
context switches and blocking are hidden by the promises, making the entire
operation fully non-blocking.

After issuing \verb!do file IO!, the NF software can continue to process other
flows. When file IO finishes, the processing of the original packet that incur
the file IO is resumed. This can greatly boost the performance of NFs that need
to perform file IO, such as PRADs.

\subsubsection{Handling Shared Variable Accessing}

\begin{verbbox}
xxx.then([]{
  access shared variable;
  return new_future;
}).then([]{
  resume packet processing;
  return new_future;
})
  
\end{verbbox}
\begin{figure}[!h]
\resizebox{0.5\columnwidth}{!}{\theverbbox}
\caption{Example code of a NF that needs to access shared variable.} 
\label{fig:asv}
\end{figure}

Figure \ref{fig:asv} shows an example of how to use promise to handle shared
variable processing. After issuing \verb!access shared variable!, the processing
is temporarily suspended and a request is created and sent to another thread to
access the shared variable. After modifying the shared variable, the suspended
packet processing is resumed.

By using promises, we can effectively linearize shared variable
accessing. Therefore we can keep the primary NF instance and the backup NF
instance in the same state, which simplifies primary-backup replication even in
multi-threaded environment.



\section{Introduction}

% nfv潮流试图将硬件middlebox替换为运行在虚拟环境中的软件middlebox,从而使关键话网络服务的部署和供给变得简单。由于软件middlebox需要被部署在关键的网络节点上,比如lte网络的骨干网络上,软件middlebox必须具有极高的性能。

The trend of Network Function Virtualization (NFV) \cite{nfv-website} aims to replace hardware middleboxes with software middleboxes running in virtualized environment. NFV greatly facilitates the deployment and provisioning of key network services. Since software middleboxes must be placed on critical network paths, such as backbone of LTE network \cite{201569}, the performance of software middleboxes must be good enough to process packets at 10/40Gbps line rate.

Besides being high performance, the ability to handle asynchronous operations is also very important to NFV. First, some middlboxes need to collaborate with each other by passing requests and responses. For instance, the middleboxes on the control plane of IMS system \cite{3gpp-ims} exchange a large number of SIP \cite{sip} requests and responses to establish a IP voice call. Second, middleboxes sometimes need to contact external service while processing flows. Stateless network function \cite{201545} stores critical flow states on an external key-value store \cite{ousterhout2015ramcloud}, for scalability and resilience. Middlebox that handle application-level protocols, like the middleboxes on the control plane of IMS system \cite{}, need to query a DNS server to identify the next-hop middlebox instance to contact with. To achieve good performance, all these middleboxes that are mentioned are preferably to handle asynchronous operations in a fully non-blocking manner.

% 为了提高软件middlebox的性能,主流的做法是跨越整个内核,并让软件middlebox直接在用户空间利用高效包处理库进行包处理。这种架构通常提供若干个工作线程直接对网卡进行繁忙轮训,以避免发送和接受包时由于内核用户空间的上下文切换而造成的巨大overhead,而这种overhead会对每秒钟需要处理几百万个包的软件middlebox产生极大的性能影响。

However, when implementing middleboxes with the state-of-art technique, non-blocking asynchronous operations can not be gracefully handled in a way that is both efficient and manageable. Today, most high-performance middleboxes bypass the kernel networking stack and process packets completely in user space with user space packet I/O frameworks, such as DPDK \cite{} and Netmap \cite{}. The user space packet I/O frameworks assign several worker threads to busily poll the network interface card (NIC) for packets, so that the overhead caused by context switches when calling traditional kernel I/O system calls is completely avoided. The use of a busy poll loop makes user-spacke packet I/O framework both efficient and easy to program with. But when incorporating asynchronous operations into the user-space packet I/O framework, a dilemma is encountered. If one would like the middlebox implementation to be easily manageable, he can replace asynchronous operations with synchronous blocking operations, but that will seriously damange the runtime performance of middleboxes, as high-performance middleboxes can not even tolerate the overhead brought by context switches. If one would like the middlebox implementation to be highly efficient, he can achieve the goal with a combination of mutable state and callback functions, but this makes middlebox implementation hard to manage, as a middlebox implementation may require to query a key-value store several times in order to process a single packet \cite{}.

%However, the use of user space I/O frameworks also limit the dominant working modes of software middleboxes to the following two, which are run-to-completion mode and pipelining mode. Run-to-completion \cite{} mode handles the fetching, processing and releasing of packets all within a single poll loop. Pipelining mode \cite{} distributes the packet processing job to multiple worker threads, each runs its own poll loop: when the poll loop of a worker thread finishes processing a packet, it hands the packet over to the next worker thread on the pipeline, or releases the packet to the outside if it is the last worker thread on the pipeline.

% 但是,这种基于用户层高性能网络报io的架构也限制了软件middlebox的工作模式,分别是run-to-completion模式和pipelining mode。run-to-completion mode在一个poll loop中完成从抓包,处理包,到把包释放出的全部过程。pipelineing mode则把包处理工作分配到多个工作线程中,当一个工作线程的poll loop完成自己的任务后,它会将包交给下一个工作线程,直到这个包被pipeline上的最后一个工作线程释放出去。这两种工作方式十分高效,并且容易编程实现。但是,这两种工作方式通常只能使用同步的方式来响应异步事件,例如进行dns查询并等待它的返回结果。这种同步的方式会极大的影响middlebox的性能。



%Both of the two working modes are highly efficient and easy to program with. However, it's hard for these two working modes to handle non-blocking asynchronous operations, i.e. perform a DNS query and wait for the response. It is trivial to integrate a blocking synchronous operation into the poll loop, but blocking synchronous operations are not efficient and will seriously damage the runtime performance of software middleboxes. It is possible to use callbacks to perform asynchronous operations, but this makes the middlebox code hard to manage, as compared to synchronous operations.



% 现有的nfv系统中也有很多响应异步事件的,例如mOS。mOS将poll loop进行了分装,并抽象出包处理过程中的事件,这些事件会被用户注册的回调函数进行响应,从而实现异步的事件处理。但是,当程序员使用mOS时,他需要使用回调函数来串接处理逻辑。和传统的工作模式相比,这种方法会打乱程序的控制流,从而使得程序员更难reason about程序。同时,mOS所暴露出的事件使基于包到达以及tcp状态机的改变的事件,这些事件并不general,无法捕捉更多工作,例如查询外部的存储设备。

There are several existing frameworks that aim to provide asynchronous event processing for middleboxes, including mOS \cite{}, libnids \cite{} and etc. These frameworks expose TCP related flow events to the middlebox programmer and enable these events to be processed in a non-blocking asynchronous manner. However, these frameworks are still based on callback-based design, making the code harder to manage when handling complex middlebox logic. They only concentrate on handling flow-level events that are related to TCP/IP protocol and they are not general enough to handle other asynchronous operations like database query.

% 在这篇论文中,我们呈现NetStar,一个为middlebox的编写提供highly scalable, lightweight异步事件处理的编程平台。NetStar基于开源事件驱动框架Seastar进行构建并使用了先进的promise-continuation编程模型。与之前的系统比较,NetStar使得middlebox可以灵活的,无阻塞的响应各种异步事件,同时也使的异步的middlebox更容易去reason,更容易实现。我们必须承认的是,为了实现高效异步事件处理,NetStar确实使用了一些抽象,但是这些抽象只会带来moderate overhead,而且NetStar仍然具有极强的多核扩展能力。

In this paper, we present NetStar, a framework for implementing middleboxes that perform non-blocking asynchronous operations. NetStar is built upon open-source asynchronous library Seastar \cite{} and provides basic constructs for building middleboxes. When compared with previous frameworks, NetStar exhibit the following advantages.

\textbf{NetStar handles asynchronous operations of middleboxes in a way that is both efficient and manageable.} Asynchronous operations in NetStar are accomplished through through callbacks, making NetStar highly efficient. However, the callbacks in NetStar are used in an implicit way that mimics the style of synchronous operations, making them easy to program with and reason about. NetStar's power comes from the promise-continuation programming model provided by Seastar \cite{} and advanced C++ features, such as lambda expression \cite{}.

\textbf{NetStar has good multi-core scalability and performance.} NetStar achieves multi-core scalability with hardware assistance. The input network traffic are automatically distributed to each core by NIC using RSS \cite{}, so that so that different cores can operate in a fully parallel fashion without contending for shared resources. NetStar is programmed with C++14 using various zero-cost abstractions. Even though the promise-continuation model that NetStar uses to implement asynchronous operations do pose slight runtime overhead, NetStar is able to compensate the overhead with multi-core scalability and achieve line-rate processing.

Using NetStar, it would be easy to implement traditional asynchronous operations such as DNS querying. Instead, we use NetStar to solve some important research problems that are raised recently.

\textbf{First,} we use NetStar to implement a non-blocking version of stateless network function \cite{}. Stateless network function \cite{} has demonstrated great potential for dynamic scaling and resilience. However, its current implementation is blocking and synchronous, making its maximum performance inferior when compared to other state-of-art middlebox systems \cite{}. We use NetStar to transform stateless network function into a non-blocking and asynchronous one and pairs it with a key-value store \cite{} with better performance than RamCloud \cite{}. Besides significantly improving the performance of stateless network function, we further identify a race problem when accessing the shared state stored on a key-value store by stateless network function. Without using NetStar, it would be hard to solve such a problem.



% 利用NetStar,我们可以很轻松的实现异步DNS查询等传统功能。为了展现NetStar的真正能力,我们着重解决最近研究界提出的几个问题:

% 实现异步无阻塞的Stateless Network Function: Stateless Network Function架构对于网络功能的容错和动态扩展都有重要价值。但是,现有的Stateless Network Function是基于一个完全同步的架构,工作线程需要对异步操作进行阻塞性等待。这使得Stateless Network Function无法有效的利用CPU资源。我们利用NetStar对Stateless Network Function进行了重写,将所有的同步等待转换成异步操作,从而极大的提高了Stateless Network Function的运行效率,同时保持了Stateless NF代码逻辑的可读性。我们也发现了Stateless Network Function在处理共享状态时的一个问题,并利用NetStar框架对其进行了有效的解决。

% 共享数据结构的访问:为了序列化共享数据结构的访问,传统NF需要对共享数据结构进行加锁。我们则利用NetSeatar框架,将访问共享数据结构的过程做成了一个消息传递过程:共享数据结构被做成了一个服务,所有工作线程对它的访问都被转换成了发送消息并接收消息访问结果。这样做保证了共享数据结构访问的线性化,同时,我们在实验中展示这样做的性能损耗并不比利用锁来处理多多少。但是这样做可以带来两个好处,首先,这样做可以解决上述提到的Stateless Network Function在处理共享状态时的多线程race问题,第二,我们可以有效记录共享数据结构的访问顺序,并有效的按照锁记录的访问顺序复制对共享数据的访问。这使得我们可以构建一个主机从机完全一致的主从复制系统,我们在实验中展示了这个系统的性能。

\textbf{Second,} we use NetStar to serialize access to shared data structures of middleboxes. Traditional middleboxes need to use lock to serialize the access to a shared data structures from multiple worker threads. We treat shared data structures as a dedicated service and convert the regular access-after-lock pattern to send-query-wait-response pattern. The new pattern is reasonably fast when compared to the access-after-lock pattern. It provides two additional benefits. First, it helps us solve the shared state access problem of stateless network function mentioned above. Second, it enables the middlebox to record the access order of any shared variables, so that another middlebox can replay the recorded access order in a deterministic fashion. We use this feature to build a primary-backup replication system for middleboxes. It's primary and back instances are always in synchronization and it has zero recovery time after primary failure.

% 作为一个探索异步操作在middlebox中实现的平台,NetStar并没有去port很多已有的middlebox。坦白来讲,由于NetStar使用了新的编程抽象,并且大量使用了C++14的语法特性,将已有的middlebox port到NetStar需要一定additional的工作量。但是,本文的重点是为了展示NetStar对支持middlebox异步操作的有效性,我们所制作的全异步Stateless Network Function相比同步的版本有重大的性能提升。其次,我们相信NetStar有足够的能力去支持更多的middlebox,NetStar基于的SeaStar平台带有内建的TCP/IP stack,这让我们有充足的理由去相信,NetStar可以支持支持mOS,因为mOS基于mTCP。

Due to the use of new programming model and massive use of C++14 feature, porting existing middleboxes to NetStar may require non-trivial amount of work. However, as a new framework for exploring how asynchronous operations can be implemented in middleboxes, porting existing middleboxes is not our primary goal. We also believe that the constructs provided by NetStar for building middleboxes are general enough to implement other middleboxes that do not perform asynchronous operations.

The paper is organized as follows. We introduce our primary motivation in section 2. The overall architecture of NetStar is given in section 3. We discuss the design of asynchronous stateless network function and serialized shared data structure access in section 4. The performance of asynchronous stateless network function and a NAT is discussed in section 5. We discuss related work in section 6 and conclude the paper in section 7.

%这篇文章的组织如下。我们在第二章介绍我们的主要motivation。在第三张,我们对NetStar框架进行一个详细介绍。在第四章,我们详细介绍异步stateless middlebox和共享数据结构的访问。我们在第五章呈现我们的evaluation结果。在第六章讨论related work。并在第七章给出论文的结论。


{\footnotesize \bibliographystyle{acm}
\bibliography{Bibliography}}


\end{document}
