\section{Rollback Recovery for Middleboxes}

\subsection{FTMB and Non-determinism in Multi-threaded NF}

FTMB \cite{sherry2015rollback} is the state-of-art middlebox replication
strategy. FTMB regularly creates a checkpoint for the primary NF instance and
generates a series of packet access logs. In the case of primary NF instance
failure, FTMB recover the primary NF instance by rollback: a new primary NF
instance is re-created based on the latest checkpoint and the state of the new
primary NF instance is recovered by replaying the packet access log. 


FTMB simply rejected the idea to keep the primary NF instance and the backup NF
instance synchronized, due to the exsitence of shared variables and
non-determinism caused by concurrently accessing shared variables.

A multi-threaded NF software may maintain several shared variables, protected by
locks. Accessing these shared variables from multiple worker threads may result in
non-determinism: when the same input sequence are fed into two identical
programs, they may fail to generate the same output.

For instance, the NF software maintain a shared variable $v$. There are two NF
instances running the same NF software. Both NF instances are configured with
two worker threads $t_1$ and $t_2$. Then, two identical packets $p_1$ and $p_2$
are concurrently fed into worker threads $t_1$ and $t_2$ respectively, on both
of the two NF instances. Because the two worker threads run in parrallel, it is
impossible to guarantee the order for accessing the shared variable $v$. We
might end-up with the following situation: on the first NF instance, $t_1$ first
accesses $v$ when processing $p_1$, while on the second NF instance, $t_2$ first
accesses $v$ when processing $p_2$. This situation may render the two identical
NF instances to be in different states.

Due to the non-determinism, the authors of FTMB \cite{sherry2015rollback} reject
the design to keep primary and backup synchronized: without tagging the packet
processing order from the master instance, it is impossible to keep the backup
instance synchronized. Even if master instance correctly tags the packet
processing order and sends these tags to the backup instance, the worker threads
on backup instance must strictly follow the packet processing order when
accessing the shared variable: before accessing a shared variable, the worker
thread must keep spinning to wait for its turn to access the shared variable,
compromising the performance of the backup instance.

\subsection{Execution Model of FTMB}

FTMB uses rollback recovery, which could be summarized as follows: the primary
instance create a \textbf{Packet Access Log}(PAL for short) whenever accessing a
shared variable. For each processed packet, the primary instance sends the
processed packet, together with all the PALs generated during processing this
packet, to the backup instance over a reliable communication channel. The backup
instance in FTMB is actually referred to as a \textbf{Output Logger}. The output
logger stores the PALs for rollback recovery and release the processed packet to
the outside. In the meantime, since the primary instance runs in a virtual
machine, the primary instance also sends a checkpoint of its virtual machine
state to the output logger. The output logger saves the current checkpoint,
discards the previous checkpoint and all the PALs received since the previous
checkpoint.

If the primary instance crashes, the primary instance is recovered as follow:
the output logger uses the saved checkpoint of the primary instance to create a
new virtual machine. Then the output logger sends the PALs and the input packets
that trigger the generation of these PALs to the new primary instance for
replay. The primary instance resumes working after it finishes replaying all the
PALs.

\subsection{Pros and Cons of FTMB}

Pros:

\textbf{Good system performance.} FTMB is carefully designed not to introduce
small overhead when generating PALs. The replicated NF in FTMB is able to
processes millions of packets per second.

\textbf{Fast recovery time.} Depending on how often the primary instance is
checkpointed, the recovery time in the case of primary failure can be as small
as 20 milliseconds.

\textbf{Passive Operation.} There is no dedicated backup instance in
FTMB. The output logger is capable of relicating multiple primary instances.

\noindent Cons:

\textbf{Packet processing latency is increased during checkpoint.} When doing
checkpoint for primary instance, the primary instance is significantly slowed
down, resulting in packet processing latency up to several milliseconds. This
leads to a huge jitter in the network RTT, which may be devastating to some
important applications such as VoIP and on-line gaming.

\textbf{The tradeoff between packet latency and checkpoint frequency.} An
important parameter to tune in FTMB is how frequently should the primary
instance be checkpointed. If the primary instance is frequently checkpointed,
then the recovery time would be faster, but each time when checkpoint is
triggered, FTMB adds huge packet processing latency, sometimes up to several
milliseconds. If the primary instance is not frequently checkpointed, then the
output logger needs a huge buffer to buffer input packets and PAL, the recovery
time is also prolonged.

\textbf{Lacking An Open-souce Solution.} FTMB is a very complicated system,
involving several critical system components that are non-trivial to
engineer. However, FTMB is not open-sourced and therefore limititing its further
application in both academia and industry.


