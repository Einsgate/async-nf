\section{Introduction}

In recent years, the research community has witnessed the quick development of
Network Function Virtualization (NFV). NFV aims to move away from hardware-based
network functions (NFs) and run software NFs in virtualized environment such as
Virtual Machines (VMs) and containers. NFV greatly simplifies the deployment of
NFs on critical network paths, as it no longer requires the network operators to
maintain proprietary hardware NFs.

%
Over the past years, the performance \cite{dpdk, rizzo2012netmap,
  martins2014clickos, hwang2015netvm}, scalability \cite{palkar2015e2} and
programmability \cite{199352, 201546} of NFV have all been greatly
improved. However, the core of NFV still lies in how to efficiently implement NF
software. Even though there are many research work \cite{199352, 201546} discussing
advancements in implementing NF software, these works are all done in an ad-hoc
manner, i.e. the advancement of these works are made using a wide range of tools
and methods, lacking a systematic framework to capture all of these
advancements.

On the other hand, there are a wide range of NFs that must be implemented for
NFV, such as firewall, NAT, asset monitor and IDS. These NFs operate on
different layers, i.e. L3-L4 for firewall and NAT, L5 for asset monitor and
IDS. Besides, NFs sometimes require more functionalities than just packet
processing, including file logging and reliable communication with a
cluster. File logging needs to deal with file system call. Reliable
communication needs a user-level TCP/IP stack. Correctly integrating
these constructs with packet processing pipeline is not a trivial task to do,
especially when considering the sequential nature of poll-based packet
processing. 

We argue that the above-mentioned problems can be well handled by proper
software abstraction and the recent advancement in high-performance server side
programming. 

To effectively integrate a wide range of NFs using a systematic framework, we
aim to rely on a widely-known asynchronous programming abstraction called
\textit{promise} \cite{vouillon2008lwt}. The promise abstraction has the two
crucial benefits for implementing future NFs. First, the promise abstraction is
expressive enough for implementing a wide range of NFs, from simple packet
processing application like firewall and NAT, to complicated NFs that analyze
the payload of a flow like asset monitor and IDS. Second, the promise
abstraction allows the programmer to write NF code in a synchronous style; the
complexity of calling asynchronous callbacks are completely hidden by the
promise abstraction.

To handle the advanced functionalities that are required by NFs, we leverage the
recent advancement in server-side programming, called Seastar \cite{seastar}. Seastar
is fully compatible with the promise abstraction. It provides high-speed
DMA-based file I/O and a user-level TCP/IP stack that is fully compliant with
the promise abstraction, making it a perfect base building block for future NF
software.

In this paper, we propose AsyncNF, which is a framework for building and porting
a wide range of NF softwares. AsyncNF provides an efficient framework for writing
NF software based on promise abstraction. To demonstrate the usability of
AsyncNF, we use AsyncNF to implement a wide range of NFs, including a
reimplementation of NetBricks, a high-speed port of PRADs asset monitor, an NF
replication scheme that does not rely on rolling-back and a simple port of
mOS. All the four NF software achieve good performance and easy
programmability.

The contributions of the paper are listed as follow:

First, we propose a framework for building NF software using promise
abstraction. The framework is capable of implementing a wide-range of NFs with
advanced functionality such as DMA-based file logging and reliable
communication.

Second, we perform four case studies using the proposed framework, demonstrating
the effectiveness of the proposed framework.
